\begin{abstract}
Brief summary of the objectives, methodology, main findings, and significance of the report. The whole story - in short.
Hvad er den (lige nu?)
\end{abstract}

% Acknowledgments
% - Acknowledge any contributions, support, or guidance from individuals, organizations, or institutions.

% TODO: Convert to IEEE format

% MSL: What is the mission? Why?
% ChemCam: How does it contribute to the mission? How?
% -> This leads us to the predictioin of composition of major oxides
% There exists a model for this, but it has some limitations
% We narrow down to limitations that in any way inhibit the precision & accuracy of the model
% // _Any_ improvements to the model will be beneficial to the mission
% We want to investigate these limitations and propose improvements

\section{Introduction}
% MSL: What is the mission? Why?
In November 2011, NASA launched the Mars Science Laboratory (MSL) mission, which landed the Curiosity rover on Mars in August 2012 inside Gale Crater. Its purpose is to investigate the Martian climate and geology, as well as the potential for life on Mars.
Early in the mission, Curiosity found chemical and mineral evidence that Mars once had the conditions necessary to support life.
As it explores Gale Crater, it acquires various samples of rock, soil, and air for onboard analysis.

The rover itself is about the size of a car.
It has various instruments and cameras, including a laser instrument called ChemCam (Chemistry and Camera), which is used to analyze the chemical composition of rocks and soil on Mars.
The primary interest are rocks that either formed in water or show indications of organic materials.

% ChemCam: How does it contribute to the mission? How?
ChemCam is a remote-sensing laser instrument developed by NASA in collaboration with the French national space agency (CNES).
It is used to determine the elemental composition of geological samples on Mars.

The ChemCam team uses a model called the Multivariate Oxide Composition (MOC) model to predict the composition of major oxides in geological samples.

% -> This leads us to the predictioin of composition of major oxides
% There exists a model for this, but it has some limitations
The MOC model is based on Partial Least Squares (PLS) regression, which is a statistical method used to model the relationship between two datasets. It also uses Independent Component Analysis (ICA) to reduce the dimensionality of the data. The model is trained on a calibration dataset consisting of 408 rock samples from Earth, which were created in a laboratory to simulate the conditions of the Martian surface.

% We narrow down to limitations that in any way inhibit the precision & accuracy of the model
% // _Any_ improvements to the model will be beneficial to the mission
% We want to investigate these limitations and propose improvements

\subsection{Problem Statement Proposal}


\section{Background}
Her vil I have missionen beskrevet?
% Background / Preliminaries (what you need to know in order to understand the story)

% What is the purpose of the MSL mission? What is the purpose of the ChemCam instrument?

The NASA ChemCam (Chemistry and Camera) is an advanced instrument developed by NASA in collaboration with the French national space agency (CNES).
It is a remote-sensing laser instrument designed to analyze the chemical composition of rocks and soil on Mars\cite{chemcam_nasa_website}.
The instrument is mounted on the Curiosity rover and is used to collect data in the Gale Crater on Mars\cite{curiosity_nasa_website}.
It leverages the ChemCam-LIBS model, which uses machine learning models to analyze the spectral data obtained from ChemCam.\cite{andersonImprovedAccuracyQuantitative2017}\cite{cleggRecalibrationMarsScience2017}

The use of LIBS (Laser-Induced Breakdown Spectroscopy) technology in planetary exploration has proven to be effective in analyzing soil and rock samples \cite{knight_characterization_2000}.

A laser pulses to ablate and remove any surface contaminants, such as dust and weathering layers, to expose the underlying material.
The laser generates a plasma plume from the now-exposed sample material.
This plasma plume emits light, which, when collected and analyzed, reveals the elemental composition of the sample by correlating the intensity of emitted light with specific wavelengths in a LIBS spectrum.
The LIBS technique enables remote analyses of materials without the need for sample preparation.
It allows for rapid analyses because of the immediate spectrum collection from the subsequent plasma, while maintaining a high spatial resolution due to its small observation footprints.
This high resolution is essential for pinpointing and investigating small features. \cite{wiens_chemcam_2012}

In 2013, \citeauthor{wiens_pre-flight_2013} published a paper describing the pre-flight calibration and initial data processing for the ChemCam LIBS instrument.
This paper introduces methods for preprocessing spectra samples and a regression model based on Partial Least Squares (PLS2) used to predict the composition of geological samples on Mars.
The model was trained on a dataset of 69 rock samples from Earth, which were created in a laboratory to simulate the conditions of the Martian surface.
This dataset is often referred to as the calibration dataset.

Two key conclusions were drawn from this paper:
\begin{enumerate}
    \item A larger dataset is needed to improve the accuracy of the model.
    \item The PLS2 model is not ideal for this type of data, and an argument is made for using PLS1 instead because of its ability to optimize each element separately, which in can improve the accuracy although it suffers slower run times.
\end{enumerate}

Based on this work, \citeauthor{cleggRecalibrationMarsScience2017} published a paper in 2017 describing a new approach to the ChemCam LIBS calibration model.
This paper introduces a new model based on PLS1 with a sub-model approach (PLS-SM) and Independent Component Analysis (ICA).
In addition, a much larger calibration dataset was used, consisting of 408 samples.
Using this, the team was able to improve the predictions by employing a \textit{submodel} PLS approach in tandem with ICA.
This model is referred to as the Multivariate Oxide Composition (MOC) model.
The MOC model is currently used by the ChemCam team to analyze the LIBS data collected by the Curiosity rover.

\subsection{Domain Background}
Describes the non-CS part of the domain/terminology that is necessary to understand the problem.
This includes relevant physics and ChemCam terminology.

Detailed explanation of the LIBS setup, including equipment, configurations, and settings.
Explain any variables, controls, and calibrations involved in the setup.

\subsection{CS Background for ChemCam calibration}
Current use of CS and statistical methods including PLS1 and ICA.
Description of how these are used in the current model.

\section{State of the Art}
Current leading approaches and methods in chemometric data analysis and prediction.
Brief description of the key technologies or techniques.

A review and analysis of the current model used by NASA, as reported in a particular paper.
Briefly address its capabilities and limitations.

Related Work (What others have done and why our method is different / novel)

Forstår ikke hvordan f.eks. Clegg et al. kan være related work, hvis det er basis for sammenligning?

\section{Methodology}
Method / Our Contribution (Detailed main part of the story, or how the analysis was done)

Hvis det er main part må I kunne tilføje flere detaljer?

\subsection{Data Analysis}
Description of the samples used and their relevance.
Explain how and why these samples were chosen.

\section{Results}
Analysis / Experimental / Empirical Evaluation (the analysis, comparing our method to SoA or a baseline, ablation study)
\begin{itemize}
    \item System used (for repeatability)
    \item Datasets / people used
    \item Metrics used
    \item Results  - hvilke resultater regner I med at have?
    \item Discussion  - skal være separat kapitel
\end{itemize}

\section{Conclusion}
Conclusion (the story again, in short, emphasizing the results)

Summary of the main findings of the report.
Reiteration of the significance of the established benchmarks to the subsequent part of the project.

\section{Recommendations for Future Work}
Suggestions for potential improvements or modifications in the methodology.
Identification of any additional benchmarks that may be relevant for future studies.
