\begin{abstract}
     Brief summary of the objectives, methodology, main findings, and significance of the report. The whole story - in short.
\end{abstract}

% Acknowledgments
% - Acknowledge any contributions, support, or guidance from individuals, organizations, or institutions.

% TODO: Convert to IEEE format

\section{Introduction}
Brief introduction to the field of study.
Importance of establishing benchmarks.
Objectives of the first part of the project.
The whole story – longer version but still short, also motivates

\section{Background}
% Background / Preliminaries (what you need to know in order to understand the story)

The NASA ChemCam (Chemistry and Camera) is an advanced instrument developed by NASA in collaboration with the French national space agency (CNES).
It is a remote-sensing laser instrument designed to analyze the chemical composition of rocks and soil on Mars\cite{chemcam_nasa_website}.
The instrument is mounted on the Curiosity rover and is used to collect data in the Gale Crater on Mars\cite{curiosity_nasa_website}.
It leverages the ChemCam-LIBS model, which uses machine learning models to analyze the spectral data obtained from ChemCam.\cite{andersonImprovedAccuracyQuantitative2017}\cite{cleggRecalibrationMarsScience2017}

The use of LIBS (Laser-Induced Breakdown Spectroscopy) technology in planetary exploration has proven to be effective in analyzing soil and rock samples \cite{knight_characterization_2000}.

Initially, a laser is fired multiple times to ablate and remove any surface contaminants, such as dust and weathering layers, to expose the underlying material.
Afterwards, the laser generates a plasma plume from the now-exposed sample material.
This plasma plume emits light, which, when collected and analyzed, reveals the elemental composition of the sample by correlating the intensity of emitted light with specific wavelengths in a LIBS spectrum.
The LIBS technique enables remote analyses of materials without the need for sample preparation.
It allows for rapid analyses because of the immediate spectrum collection from the subsequent plasma, while maintaining a high spatial resolution due to its small observation footprints.
This high resolution is essential for pinpointing and investigating small features. \cite{wiens_chemcam_2012}

In 2013, \citeauthor{wiens_pre-flight_2013} published a paper describing the pre-flight calibration and initial data processing for the ChemCam LIBS instrument.
This paper introduces methods for preprocessing spectra samples and a regression model based on Partial Least Squares (PLS2) used to predict the composition of geological samples on Mars.

% Introduce the problem - what are they actually trying to do with ChemCam, the LIBS data and their ML models?
    % Describe how ChemCam is just a part of the whole rover with a dedicated team
    % Describe how our interest/field is only a part of the ChemCam process (because instrument calibration is not part of the CS domain)
% Talk about the original pre-flight calibration paper (2013)
% Discuss the conclusions of this paper (the need for a larger dataset as well as a different approach that does not use PLS2)
% Talk about the 2017 paper (the new approach that uses PLS1 and ICA) and how it improves on the previous model
% Discuss where we are today (MOC).

\subsection{Domain Background}
Describes the non-CS part of the domain/terminology that is necessary to understand the problem.
This includes relevant physics and ChemCam terminology.

Detailed explanation of the LIBS setup, including equipment, configurations, and settings.
Explain any variables, controls, and calibrations involved in the setup.

\subsection{CS Background for ChemCam calibration}
Current use of CS and statistical methods including PLS1 and ICA.
Description of how these are used in the current model.

\section{State of the Art}
Current leading approaches and methods in chemometric data analysis and prediction.
Brief description of the key technologies or techniques.

A review and analysis of the current model used by NASA, as reported in a particular paper.
Briefly address its capabilities and limitations.

Related Work (What others have done and why our method is different / novel)

\section{Methodology}
Method / Our Contribution (Detailed main part of the story, or how the analysis was done)

\subsection{Data Analysis}
Description of the samples used and their relevance.
Explain how and why these samples were chosen.

\section{Results}
Analysis / Experimental / Empirical Evaluation (the analysis, comparing our method to SoA or a baseline, ablation study)
\begin{itemize}
    \item System used (for repeatability)
    \item Datasets / people used
    \item Metrics used
    \item Results
    \item Discussion
\end{itemize}

\section{Conclusion}
Conclusion (the story again, in short, emphasizing the results)

Summary of the main findings of the report.
Reiteration of the significance of the established benchmarks to the subsequent part of the project.

\section{Recommendations for Future Work}
Suggestions for potential improvements or modifications in the methodology.
Identification of any additional benchmarks that may be relevant for future studies.
