\section{Definition}\label{sec:definition}
Let $D$ be the LIBS data set, defined in the space $\Lambda \times \mathbb{R}^m$, where $\Lambda$ represents the set of possible wavelengths and $\mathbb{R}^m$ denotes the $m$-dimensional space of intensities.
The dataset $D$ is given by:

\begin{equation}
    D = \{ (\lambda_1, \vec{I}_1), (\lambda_2, \vec{I}_2), \ldots, (\lambda_n, \vec{I}_n) \}
\end{equation}

Each element $(\lambda_i, \vec{I}_i) \in \Lambda \times \mathbb{R}^{m}$ comprises the wavelength $\lambda_i$ of the $i^{th}$ measurement point, measured in nanometers, and an $m$-dimensional intensity vector $\vec{I}_i = [I_{i1}, I_{i2}, \ldots, I_{im}]$.
This vector captures the intensity values at $\lambda_i$ for each of the $m$ shots, measured in units of photons per channel.

Given a set of major oxides $O$ where $k=|O|$, define a model $M$ that learns a hypothesis function $f: \Lambda \times \mathbb{R}^m \rightarrow \mathbb{R}^k$ to predict the composition of the $k$ major oxides in geological samples.
The output of the hypothesis function is a vector $\mathbf{\hat{y}} = [\hat{y}_{1}, \hat{y}_{2}, \ldots, \hat{y}_{8}]$ where $\hat{y}_{i}$ is the predicted weight percentage of the major oxide $o_i \in O$.
The sum of the predicted weight percentages is not necessarily equal to 100\%.
The samples may contain other elements that are not considered major oxides, which would account for the difference.
If the sum of the predicted weight percentages is greater than 100\%, the model is overestimating the weight percentages, and represents a physical impossibility.

We define a function $E$ to measure the error of a model $M$ based on the RMSE of the predictions for each oxide $\mathbf{\hat{y}}$ and actual values $\mathbf{y}$:

\begin{equation}
    E(M) = \frac{1}{k} \sum_{i=1}^{k} \sqrt{\frac{1}{m} \sum_{j=1}^{m} (\hat{y}_{ij} - y_{ij})^2}
\end{equation}

Where \( \hat{y}_{ij} \) is the \( i^{th} \) component of the output vector \( \hat{y} \) for the \( j^{th} \) sample in the dataset \( D \), as produced by the hypothesis function \( f \) of model \( M \). Similarly, \( y_{ij} \) is the actual weight percentage of the \( i^{th} \) major oxide \( o_i \in O \) for the \( j^{th} \) sample.

Let $C = \{C_1, C_2, \ldots, C_n\}$ be the set of components that comprise a model $M$.
Given such a model and the set of its components, we define an experiment $X$ to be a change to the model $M$ that affects a subset $S(X) \subseteq C$ of its components.
Let $\mathcal{X} = \{X_1, X_2, \ldots, X_k\}$ be the set of experiments, where each $X_i$ is a function mapping from a model $M$ to a new model $M'$, defined as:

$$
X_i: M \mapsto M\underset{X_i}{\rightarrow}M'
$$

Further, associate with each experiment $X_i$ a corresponding change set $S(X_i)$:

$$
S(X_i) \subseteq C
$$

The effect of an experiment $X$ on the model $M$ results in a new model $M\underset{X}{\rightarrow}M'$ and is measured by the change in the error before and after the experiment, denoted by $\Delta E = E(M) - E(M')$.
$\Delta E > 0$ indicates an improvement in the model error, $\Delta E < 0$ indicates a deterioration in the model error, and  $\Delta E = 0$ indicates no change in the model error.
We add the model $M'$ to a set  $\mathcal{M}$, which is the set of models that result from the experiments.

This leads us to the following challenge:

\textbf{Problem}: Given a set of experiments $\mathcal{X}$ and the resulting set of models $\mathcal{M}$, identify the components $C \in M$ that contribute the most to the overall error $E(M)$. 