\section{Data Overview}\label{subsec:data_overview}
In this section, we provide an overview of the data used in this study.
The ChemCam callibration data is stored in a directory structure as shown in Figure~\ref{fig:directory_structure}.

\begin{figure}
    \begin{forest}
        for tree={
            font=\ttfamily,
            grow'=0,
            child anchor=west,
            parent anchor=south,
            anchor=west,
            calign=first,
            inner xsep=7pt,
            edge path={
                \noexpand\path [draw, \forestoption{edge}]
                (!u.south west) +(7.5pt,0) |- (.child anchor) pic {folder} \forestoption{edge label};
            },
            % style for your file node 
            file/.style={edge path={\noexpand\path [draw, \forestoption{edge}]
                (!u.south west) +(7.5pt,0) |- (.child anchor) \forestoption{edge label};},
                inner xsep=2pt,font=\small\ttfamily
                            },
            before typesetting nodes={
                if n=1
                {insert before={[,phantom]}}
                {}
            },
            fit=band,
            before computing xy={l=15pt},
            }  
        [calib\_2015
            [1600mm
                [ica
                    [0.1tio2
                        [2015\_03\_27\_132008\_ccs.csv, file]
                        [2015\_03\_27\_132008\_ccs.lbl, file]
                        [2015\_03\_27\_132210\_ccs.csv, file]
                        [2015\_03\_27\_132210\_ccs.lbl, file]
                        [2015\_03\_27\_132331\_ccs.csv, file]
                        [2015\_03\_27\_132331\_ccs.lbl, file]
                        [2015\_03\_27\_132453\_ccs.csv, file]
                        [2015\_03\_27\_132453\_ccs.lbl, file]
                        [2015\_03\_27\_132624\_ccs.csv, file]
                        [2015\_03\_27\_132624\_ccs.lbl, file]
                    ]
                    [1tio2]
                    [$\vdots$, file]
                    [wc1]
                    [wc3]
                ]
                [pls
                    [0.1tio2]
                    [1tio2]
                    [$\vdots$, file]
                    [wc3]
                    [wm]
                ]
                [ccam\_calibration\_compositions.csv, file
                ]
            ]
        ]
    \end{forest}
\caption{Directory structure of the data.}
\label{fig:directory_structure}
\end{figure}

As shown in the figure, the data used to train the ICA and PLS models are stored in two different directories, \texttt{ica} and \texttt{pls}, respectively, however many of the datasets are shared between the two models.
For each sample, the data is split into five datasets, one for each location on the sample that was shot at by the laser.
Each dataset contains CCS data stored in a \texttt{.csv} file and a corresponding \texttt{.lbl} file containing metadata.
We do not use the metadata in the \texttt{.lbl} files in this study.

Each \texttt{.csv} file contains the following columns:

\begin{itemize}
    \item \texttt{wave}: The wavelength of the spectral data measured in nanometers (nm).
    \item \texttt{shot1}, \texttt{shot2}, ..., \texttt{shot50}: The intensity measurement for each wavelength at the corresponding shot measured in photons/pulse/mm²/sr/nm.
    \item \texttt{median}: The median of the intensity measurements for each wavelength.
    \item \texttt{mean}: The mean of the intensity measurements for each wavelength.
\end{itemize}

Table \ref{fig:ccs_data_example} shows an example of the CCS data for the first sample in the \texttt{0.1tio2} directory.

\begin{table*}[ht]
\centering
\begin{tabular}{rrrrrrrr}
\toprule
     wave &        shot1 &      shot2 &      $\cdots$ &       shot49 &       shot50 & median & mean \\
\midrule
241.66901 & 2.820443e+11 & 1.3349191e+16 & $\cdots$ & 9.3383243e+15 & 9.4850282e+15 & 9.6073916e+15 & 1.0412862e+16 \\
241.72301 & 3.279678e+11 & 3.9035181e+12 & $\cdots$ & 3.5982364e+12, & 4.7538387e+12 & 2.9107688e+12 & 3.2675139e+12 \\
$\vdots$  & $\vdots$     & $\vdots$ & $\cdots$ & $\vdots$ & $\vdots$ & $\vdots$ & $\vdots$ \\
905.38062 & 9.2772428e+09 & 1.1861747e+10 & $\cdots$ & 3.4135089e+09 & 2.8770024e+09 & 6.7861770e+09 & 1.7767384e+09 \\
905.57349 & 1.9527751e+10 & 2.4984806e+10 & $\cdots$ & 7.1648592e+09 & 6.0560959e+09 & 1.4299386e+10 & 2.7702141e+09 \\ 
\bottomrule
\end{tabular}
\caption{Example of CCS data for the first sample in the \texttt{0.1tio2} directory.}
\label{tab:ccs_data_example}
\end{table*}
