\section{Proposed Approach}
To address the challenges in predicting major oxide compositions from \gls{libs} data, we propose the development of advanced computational models capable of effectively handling the multifaceted challenges we describe in \ref{subsec:challenges}.
These issues complicate the accurate and robust prediction of elemental concentrations, necessitating advanced computational methodologies. 

Our approach aims to enhance the prediction accuracy and robustness for major oxides in \gls{libs} data by leveraging specific combinations of machine learning models and preprocessors that are particularly effective at predicting individual oxides.
The models will use feature vectors $\mathbf{x} \in \mathbb{R}^N$ derived from the Masked Intensity Tensor $\mathbf{M}[\chi, l, \lambda]$ as input, where $N$ is the number of features. 
The output will be Estimated Concentration Vectors $\mathbf{v} \in \mathbb{R}^{n_o}$.

As the literature highlighted in Section~\ref{sec:related-work} suggests, a variety of models and preprocessing techniques promise to be adept at handling data that exhibit high-dimensionality, multi-collinearity, and matrix effects.
The literature also indicates that different machine learning models perform better on some oxides than others.
These challenges and model-specific strengths suggests that an optimal approach would involve hybrid methodology, integrating multiple models and preprocessing steps tailored to the specific characteristics of the data.
This could include leveraging ensemble learning techniques to combine the predictions of various models, implementing dimensionality reduction techniques like \gls{pca} to mitigate high-dimensionality issues, and employing robust preprocessing strategies to address multi-collinearity and matrix effects.
Furthermore, a systematic evaluation through cross-validation and hyperparameter tuning would be essential to fine-tune the models for the best performance on the specific oxides of interest.
The notion of using multiple models per oxide is supported by the advent of models such as the \gls{moc}~\cite{cleggRecalibrationMarsScience2017} model, which combines the predictions of multiple models using a predetermined weighting for each model's predictions on a per-oxide basis.
While this approach improved accuracy compared to individual models, it required manual tuning of the weights for each model.
This manual tuning presents limitations, including the analysis required to determine appropriate weights and the risk of suboptimal weighting.

Given these limitations, it is reasonable to explore techniques that can automate the weighting process while still leveraging the strengths of multiple models.
To address these criteria, we chose to adopt a stacking ensemble approach. 
Stacking, as described in Section~\ref{subsec:stacked-generalization}, is a method that utilizes multiple base estimators trained on the same data, whose predictions are then used to train a meta-learner.
By combining a diverse set of base models, stacking can correct for the biases of individual models.
Since each model focuses on different patterns within the data, stacking mitigates the inherent biases of individual models by estimating and correcting for these biases.
This approach of leveraging the strength of multiple models that each model the problem differently can lead to better generalization on unseen data by automating and potentially improving upon manual tuning through the use of a meta-learner to discern patterns in the base predictors' outputs. \cite{wolpertstacked_1992} \cite{survey_of_ensemble_learning}
However, some consideration has be made towards training of the base models in order to prevent data leakage and overfitting.
As emphasized by \citet{cvstacking}, if the base models are trained on the same dataset, the meta learner might favor certain base models over others.
This can cause the meta learner to be influenced by the same patterns and biases that the base models are susceptible to, leading to overfitting.
To mitigate this risk and ensure generalizability, a cross-validation strategy should be employed to ensure that the meta learner's training data accurately reflects the true performance of the base learners.

We adopted an experimental approach to empirically evaluate the potential of various models and preprocessing techniques, to be used in our stacking ensemble, ensuring that our selections were informed by our literature review while also allowing for independent assessment and validation.

To systematically address the challenges in predicting major oxide compositions from \gls{libs} data, we have devised a comprehensive approach that integrates model and preprocessing selection, an experimental framework, evaluation and comparison, and the construction of a stacking ensemble.

Firstly, we conducted an extensive literature review and preliminary experiments to select a diverse set of machine learning models and preprocessing techniques.
These include ensemble learning models, linear and regularization models, neural network models, scaling methods, dimensionality reduction techniques, and data transformations.
This selection process is detailed in Section~\ref{sec:model_selection}.

Next, we implemented an experimental framework using the Optuna optimization library~\cite{optuna_2019}.
This framework facilitates automated hyperparameter optimization, allowing us to efficiently explore a vast search space of model and preprocessing configurations.
The specifics of this framework are discussed in Section~\ref{sec:optimization_framework}.

Each configuration is then evaluated using a customized k-fold cross-validation procedure to ensure robust performance assessment, as detailed in Section~\ref{subsec:validation_testing_procedures}.
This procedure addresses challenges such as data leakage and uneven distribution of extreme values.
The results from these evaluations are compared to identify the best-performing configurations for each oxide.

Furthermore, we present the metrics we will use to evaluate the performance of our models in Section~\ref{subsec:evaluation_metrics}.
These metrics include the \gls{rmse} for accuracy and the sample standard deviation of prediction errors for robustness.
By evaluating both cross-validation and test set metrics, we ensure a comprehensive assessment of the model's generalizability and performance on unseen data.

Finally, the top-performing configurations are used to construct a stacking ensemble.
This ensemble leverages the strengths of multiple models, with a meta-learner trained to optimize the final predictions.
The process of constructing and validating this stacking ensemble is described in Section~\ref{sec:evaluation_metrics}.

By following this structured approach, we aim to enhance the prediction accuracy and robustness for major oxides in \gls{libs} data, ultimately leading to more reliable and generalizable models.

\subsection{Model and Preprocessing Selection}

Choosing the right models and preprocessing techniques for \gls{libs} data analysis is a challenging task. 
As the literature highlighted in Section~\ref{sec:related-work} suggests, a variety of models and preprocessing techniques promise to be adept at handling data that exhibit high-dimensionality, multi-collinearity, and matrix effects.
The literature also indicates that different machine learning models perform better on some oxides than others.
These challenges and model-specific strengths suggests that an optimal approach would involve combining multiple models. 
This notion is supported by the advent of models such as the \gls{moc}~\cite{cleggRecalibrationMarsScience2017} model, which combines the predictions of multiple models using a predetermined weighting for each model's predictions on a per-oxide basis.
While this approach improved accuracy compared to individual models, it required manual tuning of the weights for each model.
This manual tuning presents limitations, including the analysis required to determine appropriate weights and the risk of suboptimal weighting.
Given these limitations, it is reasonable to explore techniques that can automate the weighting process while still leveraging the strengths of multiple models.
To fulfill these criteria, we chose to adopt a stacking ensemble approach. 
Stacking, as described in Section~\ref{subsec:stacked-generalization}, is a method that utilizes multiple base estimators trained on the same data, whose predictions are then used to train a meta-learner.
By combining a diverse set of base models, stacking can correct for the biases of individual models.
Since each model focuses on different patterns within the data, stacking mitigates the inherent biases of individual models by estimating and correcting for these biases.
This approach of leveraging the strength of multiple models that each model the problem differently can lead to better generalization on unseen data by automating and potentially improving upon manual tuning through the use of a meta-learner to discern patterns in the base predictors' outputs. \cite{wolpertstacked_1992, survey_of_ensemble_learning}
However, some consideration has be made towards training of the base models in order to prevent data leakage and overfitting.
As emphasized by \citet{cvstacking}, if the base models are trained on the same dataset, the meta learner might favor certain base models over others.
This can cause the meta learner to be influenced by the same patterns and biases that the base models are susceptible to, leading to overfitting.
To mitigate this risk and ensure generalizability, a cross-validation strategy should be employed to ensure that the meta learner's training data accurately reflects the true performance of the base learners.

We adopted an experimental approach to empirically evaluate the potential of various models and preprocessing techniques, to be used in our stacking ensemble, ensuring that our selections were informed by our literature review while also allowing for independent assessment and validation.

We had several considerations to guide our selection of preprocessing techniques.
Firstly, our review of the literature revealed that there seems to be no consensus on a single, most effective normalization method for \gls{libs} data.
This led us to include the traditional normalization methods, such as z-score normalization, Min-Max scaling, and Max Absolute scaling, in our experiments.
This approach allowed us to determine which normalization method was most effective for our dataset. 
Additionally, dimensionality reduction techniques are considered by the literature to be effective techniques for \gls{libs} data due to its high dimensionality. Specifically, \gls{pca} has been widely adopted by the spectroscopic community as an established dimensionality reduction technique~\cite{pca_review_paper}. However, \citet{pca_review_paper} makes the case that the assumptions for \gls{pca} regarding linearity of the data are only met up to a certain point, after which it breaks. They argue that this non-linearity inherent in the data makes \gls{kernel-pca} a valid candidate for \gls{libs} data. Based on their review of the field, and our own review of the literature, not many have studied the effectiveness of \gls{kernel-pca} in the context of \gls{libs} data. Therefore, we decided to include this in our experiments to further assess its potential. In addition to the non-linearity, \citet{pca_review_paper} also argue that the assumptions of normality in the data are not always met in \gls{libs} data. For this reason, we decided to include power transformation and quantile transformation in our experiments as models such as \gls{pca} benefit from a normal distribution of the data. We assume that models such as \gls{pls} may also benefit from a more Gaussian-like distribution of the data, by virtue of the model being partly based on \gls{pca}.

While these preprocessing techniques are not an exhaustive list, they represent a diverse set of methods.
Techniques such as feature selection were not considered, in order to limit the scope of the study and due to time constraints.

We also had several requirements for the selection of models.
The models selected for experimentation had to be diverse to ensure that we had enough breadth in our results to make informed decisions about which models should be included in the final stacking ensemble pipeline.
Additionally, the models had to be suitable for regression tasks and if no research was available in the context of \gls{libs} data, they had to be models that have shown promise in other domains.
Our literature review found that a variety of models fit this criteria.
For example, \citet{andersonPostlandingMajorElement2022} demonstrated that that models such as \gls{gbr}, \gls{pls}, \gls{lasso}, \gls{rf} were each effective at predicting different major oxides.
Additionaly, \citet{svrforlibs} demonstrated in their work that \gls{svr} outperforms \gls{plsr} in predicting \ce{Si}, \ce{Ca}, \ce{Mg}, \ce{Fe} and \ce{Al}.
As a result, we have included \gls{gbr}, \gls{pls}, \gls{lasso}, \gls{rf}, and \gls{svr} in our experiments.

In the neural network domain, \cite{ann_libs_soil_analysis} showed that on \gls{libs} data their 3-layer \gls{ann} relative error of prediction for \ce{Ca}, \ce{Fe}, \ce{Al} was below 20\%.
Likewise, \citet{yangConvolutionalNeuralNetwork2022} showed that \gls{cnn} outperformed methods such as \gls{logreg}, \gls{svm} and linear discriminant analysis at correctly classifying twelve different types of rocks based on \gls{libs} data. 
While this example for \gls{cnn} is a classification task, \gls{cnn} can be adjusted for regression tasks by changing the loss function and output layer.
Based on these factors, we decided to include \gls{ann} and \gls{cnn} in our experiments to further increase the diversity of our model selection.

To further bolster our selection pool, we chose to include models that were in the same family of models as those that showed promise in the literature.

\gls{xgboost} was included as an option based on its promising accuracy in various settings.
For example \citet{xgboost_in_biomedicie} showed that \gls{xgboost} outperformed models such as \gls{rf} and \gls{svm} in predicting biological activity based on quantitative description of the compound's molecular structure. 
Another example is \citet{xgboost_in_heart_disease} that used \gls{xgboost} to predict heart disease, where \gls{xgboost} outperformed \gls{rf} and \gls{etr} at correctly classifying patients with heart disease.
Due to these factors and the limited study of \gls{xgboost} in the context of \gls{libs} data, we decided to include it in our experiments.

Following the same logic, we included \gls{ngboost}. 
\gls{ngboost} is a recent model introduced by \citet{duan_ngboost_2020} that according to their work improves upon \gls{gbr} by using a more sophisticated loss function and a more advanced gradient boosting algorithm.
A very limited amount of research has been conducted using this algorithm in the context of \gls{libs} data, however, \citet{ngboost_landslide} showed that \gls{ngboost} outperformed \gls{xgboost} and \gls{rf} and correctly determining landslide-prone fields with an AUC of 0.898 compared to an AUC of 0.871 and 0.863 of \gls{xgboost} and \gls{rf} respectively.

Finally, \gls{ridge}, \gls{enet}, \gls{etr} and \gls{lasso} were used in various studies and showed promising results, despite not necessarily being best performing in their respective studies.
Therefore, we chose to include these in our experiments to further diversify our model selection. 

Table~\ref{tab:preprocessing-models} summarizes the preprocessing techniques and models selected for our experimentation.

\begin{table}[ht]
\centering
\begin{tabularx}{\columnwidth}{>{\raggedright\arraybackslash}X}
\toprule
\textbf{Normalization / Scaling:} \\
\midrule
Z-Score Normalization \\
Min-Max Normalization \\
Max Absolute Scaling \\
Robust Scaling \\
Norm 3 \\
\midrule
\textbf{Transformation Methods:} \\
\midrule
Power Transformation \\
Quantile Transformation \\
\midrule
\textbf{Dimensionality Reduction Methods:} \\
\midrule
PCA \\
Kernel PCA \\
\midrule
\textbf{Model Types:} \\
\midrule
\textbf{Regression Models:} \\
\quad Partial Least Squares \\
\quad Support Vector Regression \\
\quad Elastic Nets \\
\quad Least Absolute Shrinkage and Selection Operator \\
\quad Ridge Regression \\
\textbf{Ensemble Models:} \\
\quad Random Forest \\
\quad Gradient Boost Regression \\
\quad Extra Trees Regression \\
\quad XGBoost \\
\quad Natural Gradient Boosting \\
\textbf{Neural Networks:} \\
\quad Artificial Neural Networks \\
\quad Convolutional Neural Networks \\\bottomrule
\end{tabularx}
\caption{Overview of Preprocessing Techniques and Models}
\label{tab:preprocessing-models}
\end{table}
\subsection{Optimization Framework}\label{sec:optimization-framework}
One of the primary challenges in developing a stacking ensemble is determining the optimal choice of base estimators.
\citet{wolpert1992stacked} highlighted that this can be considered a 'black art' and that the choice usually relies on intelligent guesses.
In our case, this problem is further exacerbated by the fact that the optimal choice of base estimator may vary depending on the target oxide and the fact that we are also considering the optimal preprocessing techniques for each base estimator.
This means, that we are considering an entire pipeline of preprocessing techniques and base estimators for each target oxide.
Therefore, we need a systematic approach to determine the optimal configuration for each pipeline.
To guide this process we have developed a working assumption.
Namely, that we assume that selecting the top-$n$ best pipelines for each oxide, given different preprocessors and models per pipeline, will yield the best pipelines for a given oxide in our stacking ensemble.
Here, $n$ is a heuristic based on the results and \textit{best} is evaluated in terms of the metrics outlined in Section~\ref{sec:evaluation_metrics}.
This reduces the problem to finding a selection of preprocessors and models, and a method of testing multiple permutations of these to find the top-$n$ best pipeline for each oxide.
In Section~\ref{sec:model-selection} we outlined the models and preprocessing techniques that we intend to use and in the following section we will describe the optimization framework that we have developed to address this challenge.
% To address this challenge we have developed a working assumption, namely; that the top-n best performing pipelines for each oxide, given different preprocessors and models per pipeline, are the best pipelines for a given oxide in our stacking ensemble.
% How to determine the optimal configuration for predictor (model & preprocessors) that we can use in our stacking pipeline?
% Assumption: finding the top-n best predictors, given different models per predictor, is assumed to be the best configuration per-oxide for our stacking pipeline.
% Maybe call back to original stacking paper (Wolpert) regarding there not being a tried-and-true method for determining the best models for a stacking approach
% What about doing a search that checks as many configurations as feasible for their performance?! ← Our approach
% But you'd still need an initial list of models and preprocessors to consider (← lit review + initial experiments – model selection callback)


















Optuna is a hyperparameter optimization framework designed for direct integration with Python. Its dynamic embedding capability allows hyperparameters to be defined and adjusted within the code during execution, providing additional flexibility and ease of debugging. 
Optuna uses advanced sampling strategies to explore promising areas of the search space and employs pruning techniques to terminate unpromising trials early, optimizing computational resource use. 
It also supports scaling across multiple machines by distributing the optimization process, allowing for concurrent optimization. 
Additionally, tools are provided to identify and focus on the most impactful parameters, aiding in the efficient handling of high-dimensional search spaces. \cite{optuna_2019} 
Because an essential aspect of our study is to identify the optimal configuration of model and preprocessing techniques for that model on a per-oxide basis, these factors make Optuna an ideal choice as the basis of our optimization framework.
Optunas flexibility meant that it was easy to customize the optimization process to solve this problem and integrate it with our existing codebase.

At the core of the framework is what is referred to as the objective function.
The objective function acts as the primary building block, where the sampling parameters are set for the preprocessing techniques and the models and is the function that is minimized. % rephrase
Sampling parameters in this context relate to the range of values or options that the optimizer can choose from.

Because we want to identify not only the best model configuration, but also the best preprocessing configuration for that model, we have combined the sampling of the model and preprocessing parameters into a single objective function.
By doing this, we can optimize the entire pipeline in a single step, rather than optimizing the model and preprocessing steps separately.
% We 
With such a setup, Optuna will attempt to optimize any permutation of scaler, transformer, dimensionality reduction, and model hyperparameters throughout the optimization process.
This allows us to fully utilize Optunas sampling and pruning strategies, minimizing wasted time and computational resources on poor configurations.

\begin{algorithm}
\caption{Hyperparameter Optimization Framework}
\label{alg:hyperparameter_optimization_framework}
\begin{algorithmic}[1]
\Require Dataset $D$, Model $m$, Number of Trials $N$, Random Seed $seed$, Sampler \texttt{sampler}, Pruner \texttt{pruner}
\Ensure Trial data, including metrics and configuration for each trial

\State \textbf{Initialize:} Set random seed for reproducibility if seed is not None \label{step:initialize}
\For{each trial $t$ from 1 to $N$} \label{step:trial_loop}
    \State $hp \gets \texttt{sample\_hyperparameters}(m, \texttt{sampler})$
    \State $m' \gets \texttt{instantiate\_model}(m, hp)$ \label{step:instantiate_model}
    \Statex
    \State $s\_params \gets \texttt{sample\_scaler\_params}(\texttt{sampler})$ \State $s \gets \texttt{instantiate\_scaler}(s\_params)$
    \State $t\_params \gets \texttt{sample\_transformer\_params}(\texttt{sampler})$
    \State $t \gets \texttt{instantiate\_transformer}(t\_params)$ \newline \hspace*{3em} \textbf{or} NONE
    \State $dr\_params \gets \texttt{sample\_dim\_reduction\_params}(\texttt{sampler})$
    \State $dr \gets \texttt{instantiate\_dim\_reduction}(dr\_params)$ \newline \hspace*{3em}\textbf{or} NONE
    \Statex     
    \State $pipeline \gets [s, t, dr]$ \textbf{or} $[s, NONE, NONE]$
    \State $T_{cv}, D_{train}, D_{test} \gets \text{apply data partitioning to } D$
    \Statex
    \State $D_{train}'$ $\gets$ \texttt{pipeline\_apply}($D_{train}$, \texttt{pipeline})
    \State $T_{cv}'$ $\gets$ \texttt{pipeline\_apply}($T_{cv}$, \texttt{pipeline})
    \Statex
    \State $CV_{metrics} \gets \texttt{cross\_validate}(m', T_{cv}')$
    \State $rmse_{cv} \gets \texttt{average}(CV_{metrics}.\texttt{rmse\_values})$
    \State $std\_dev_{cv} \gets \texttt{average}(CV_{metrics}.\texttt{std\_dev\_values})$
    \Statex
    \State $m'_{train}$ $\gets$ \texttt{train}($m'$, $D_{train}'$)
    \State $rmsep$, $\sigma_{error, test}$ $\gets$ \texttt{evaluate}($m'_{train}$, $D_{test}$)
    \Statex
    % function that stores the metrics
    \State \texttt{store\_metrics}($t$, $m'$, $pipeline$, $rmse\_cv$, \newline \hspace*{8em}$std\_dev\_cv$, $rmsep$, $\sigma_{error, test}$)
\EndFor
\State \Return \texttt{rmse\_cv}, \texttt{std\_dev\_cv}, \texttt{rmsep}, $\sigma_{error, test}$
\end{algorithmic}
\end{algorithm}% This increases the likelyhood at identifying the 

% create names for evaluation metrics
% 



% Optuna for its flexibility and efficiency in exploring the vast search space of configurations.
% Optuna allows for great flexiblity in exploring various search spaces due to its modular design.
% In addition to this modularity, Optuna uses a "define-by-run" optimization strategy.
% Rather than being confined to a fixed order and range of exploration, Optuna can dynamically adjust regions based on the results of previous trials.
% Introduction of the optimization framework
    % What is the goal of the optimization framework?
    % Why Optuna?
        % Explanation of what optuna is and why it is useful for our purposes
        % Should elaborate on this:
            % This framework facilitates automated hyperparameter optimization, allowing us to  efficiently explore a vast search space of model and preprocessing configurations.
% how did we do it?
    % Maybe talk about how we designed it in this way
    % pros cons ** maybe
    % Considerations at the very least
    % Talk about how the objective function was constructed, the order of things, considerations with this, the fact that we are trying to minimize and what metrics we are minimizing
    % 
% show diagram or pseudocode
% Explain why we did it this way
    % Why did we choose this approach?
    % What are the benefits of this approach?
    % How does this approach help us achieve our goal?

   
    
% Next, we implemented an experimental framework using the Optuna optimization library~\cite{optuna_2019}.
% This framework facilitates automated hyperparameter optimization, allowing us to efficiently explore a vast search space of model and preprocessing configurations.
% The specifics of this framework are discussed in Section~\ref{sec:optimization_framework}.


% Optuna uses python syntax instead of some propriety syntax
% Optuna brings the optimization into the space of the program rather than outside
    % Meaning you can embed it directly in functions etc.
    % Easier to debug
    % Can use python language such as looping etc.
% Instead of having parameters set you change them so they are sampled using the suggest_* methods
    % This is also helpful because it allows you to optimize any parameter, even for preprocessing etc. in the same objective function
% There are two parts of the hyperparameters optimizer: Sampling strategy and Pruning strategy
    % Sampling strategy determines where to look
        % Uses bayesian filtering to find places where it has had the best results and focus in on those
        % As Optuna tries to minimize/maximize the objective function it focuses in on the best areas and makes more trials there
        % There are multiple samplers - Even the traditional ones
        % Choosing the right sampler is sort of a heuristic 
    % Pruning strategy can terminate trials early that are not promising so that the compute can be dedicated to more promising trials
        % Basically trials that have a slow start and will never be able to make up for that slow start, are pruned
% Optuna is very easy to scale up. It allows you to use a single database across multiple machines
    % This means that Optuna will let you use multiple computers to optimize over the search space at the same time
    % Its called asynchronous parallelization of trials - One trial could start later than the other
    % Near linear scaling with the number of machines
% Has tools to help you determine most important parameters to avoid curse of dimensionality
    % get_param_importances
    % It works by running a small number of trials and then returns this information to you
    % Helps you dial in where optuna should be focusing to get most optimal results

\subsection{Validation and Testing Procedures}
This section describes the validation and testing procedures that each of our experiments adhere to.
While we use conventional techniques like holdout sets and k-fold cross validation, \gls{libs} impose additional challenges to the process.

One of the primary challenges lies in ensuring there is no data leakage.
Therefore, we group shots on a given location for a target.
As per matrix A in Section~\ref{sec:problem_definition}, we know that each target only has one ground truth concentration value per oxide.
Since the emitted light intensities varies slightly for each location that is shot at on the target, we essentially have $l$ (locations per target) entries for the same ground truth value.
This means that if we randomly split the dataset, some locations from a target would end up in the testing set, while others in the training set.
This would result in data leakage, as the testing set would no longer be representative of the overall dataset.

% TODO: We'll be rewriting part of this due to rewriting our approach to splitting. We'll emulate the method employed by Anderson et al. completely, meaning we'll do this:
% > "To assess the performance of the PLS model, we use k-fold cross validation [55] and a separate test-set. We divided the full set of laboratory data into five folds, four of which are used for cross validation and are combined as the final training set. The fifth fold is withheld and used as a test set to validate the final model. Folds were defined separately for each element, after removal of outliers. To ensure that each fold represented the full elemental compositional variation, the samples were sorted based on the major element of interest, and samples were then assigned sequentially to each fold. In some cases, the database contains only a small number of targets with very high or very low compositions for a given element. These extreme targets can have a strong influence on the model and enable it to handle a wider range of compositions, so they are forced to be in the training folds rather than the test set. Unless otherwise specified, the results presented are the test set results."


\subsubsection{Dataset Partitioning}
To ensure a rigorous evaluation of the models, we first split the dataset into training and testing subsets.
As described in Section~\ref{sec:baseline_replica}, we adopted an automated approach to identify and distribute extreme compositions evenly across both subsets.
This method was devised to replicate the dataset splitting methodology employed by \citet{andersonImprovedAccuracyQuantitative2017}, as we have described in \citet{p9_paper}.
The process of dataset partitioning involves several key steps aimed at ensuring that extreme values are appropriately represented, which is critical given the skewness that such values can introduce into the training process:

\begin{enumerate}
    \item \textbf{Identification of Extremes:} For each oxide in the dataset, the samples are sorted by concentration. The extremes, defined as the $n$ highest and lowest values, are identified for inclusion in the training set.
    \item \textbf{Separation of Extremes:} These extreme samples are first segregated to ensure they are not included in the subsequent random partitioning. This step guarantees that the training set contains critical outliers which are often informative for model robustness.
    \item \textbf{Random Partitioning:} The remainder of the dataset, excluding the previously segregated extremes, undergoes a random split. The splitting process adheres to a predefined ratio, typically set at 80\%/20\% for training and testing, respectively.
    \item \textbf{Reintegration of Extremes:} The identified extremes are reintegrated into the training subset. This methodological step ensures that the training data encompasses a comprehensive range of the data's variability, particularly the tail-end characteristics that are crucial for robust model performance.
    \item \textbf{Adjustment of Test Size:} Given the inclusion of extremes in the training set, the proportion of the dataset allocated for testing is adjusted accordingly. This adjustment ensures that the overall ratio of training to testing remains as intended, despite the prior allocation of extreme samples to the training set.
\end{enumerate}

This approach not only aids in achieving a balanced dataset but also in maintaining the integrity of the testing process by avoiding any potential leakage of information between the training and testing datasets.

% Need to discuss our reasoning behind selecting n=2 for separating extreme values. Should be backed by data analysis ideally.

As previously discussed, we have opted for an 80\%/20\% division between the training and testing datasets.
This ratio is strategically chosen to maximize the training set's capacity for effective model learning while ensuring the testing set is sufficiently representative to provide an accurate assessment of the model's performance on new, unseen data.
Expanding the testing set beyond this proportion is not recommended.
As detailed in Section~\ref{sec:problem_definition}, one of the primary constraints we face is the limited availability of data.
Allocating too much data to the testing set could compromise the comprehensiveness of the training set, potentially undermining the model's ability to generalize from a robust learning process.


\subsubsection{Cross-Validation Strategy}
In this project, we implement a robust mechanism to prevent our model from merely tuning to peculiarities of a specific dataset segment.
Traditional approaches, where models are validated against a singular test set, might inadvertently result in models that perform well on that set but poorly generalize to new data.
To overcome this, we utilize a k-fold cross-validation strategy, which is particularly designed to enhance model generalizability across various unseen data scenarios.

Our strategy adopts a group-based variant of k-fold cross-validation to address potential data leakage, which can occur when closely related data points are scattered across both training and testing sets.
This can mislead the evaluation of the model's performance.
To mitigate this, our method ensures that all measurements related to a single entity (as defined by a grouping attribute) are contained entirely within either the training set or the testing set, but not both.

% What is a group?

The custom cross-validation method is delineated in Algorithm \ref{alg:custom_k_fold}: 

\begin{algorithm}[H]
\caption{Custom K-Fold Cross-Validation}
\label{alg:custom_k_fold}
\begin{algorithmic}[1]
\Require Dataset $D$, Number of folds $k$, Grouping attribute \textit{group\_by}, Random seed \textit{random\_state}
\Ensure Sequence of training and testing datasets for each fold

\State Group $D$ by \textit{group\_by} into $G$ \label{line:group}
\State Extract unique keys from $G$ into \textit{keys} \label{line:extract_keys}
\State Shuffle \textit{keys} using \textit{random\_state} \label{line:shuffle}
\State Split \textit{keys} into $k$ folds using K-Fold technique \label{line:split}
\For{$i = 1$ to $k$}
    \State Select the $i$-th fold as test keys, and the rest as train keys \label{line:select_keys}
    \State Concatenate groups corresponding to train keys to form \textit{train\_data} \label{line:concatenate_train}
    \State Concatenate groups corresponding to test keys to form \textit{test\_data} \label{line:concatenate_test}
    \State Yield $(\textit{train\_data}, \textit{test\_data})$ \label{line:yield}
\EndFor
\end{algorithmic}
\end{algorithm}

Initially, the dataset \(D\) is grouped by a specified attribute, resulting in groups \(G\) as described in line \ref{line:group}.
Unique group identifiers are extracted into \(keys\) (line \ref{line:extract_keys}), which are then shuffled (line \ref{line:shuffle}) to ensure randomness, utilizing a provided random seed \textit{random\_state}.

These keys are divided into \(k\) folds (line \ref{line:split}), and for each iteration from 1 to \(k\), one fold is selected as the test set, with the remaining serving as the training set (line \ref{line:select_keys}).
Corresponding data for each set of keys is then aggregated to form the training data (\textit{train\_data}) and the testing data (\textit{test\_data}), respectively, as indicated in lines \ref{line:concatenate_train} and \ref{line:concatenate_test}.
This ensures that all data from any single group is exclusively included in either the training or the testing set, thus mitigating the risk of data leakage.

This custom cross-validation strategy is crucial for ensuring that our evaluations are as realistic as possible, providing a reliable estimate of how the model will perform on truly independent data.
Through this method, we enhance the likelihood that our model's effectiveness is genuine and not a result of overfitting to the idiosyncrasies of the test data.


\subsubsection{Evaluation Metrics}
As mentioned in Section~\ref{sec:problem_definition}, the performance of the models was quantitatively assessed using the \gls{rmse} and the standard deviation of the residuals.
These metrics are calculated for each fold and averaged across all folds to provide comprehensive indicators of model accuracy and variability.
In addition, we also compute the metrics for the test set to provide a measure of the model's performance on unseen data.
Therefore, we have the following metrics for each experiment:
\begin{enumerate}
    \item \textbf{Fold-specific RMSE and Standard Deviation:} For each of the $k$ folds, we calculate both the RMSE and standard deviation, denoted as \texttt{rmse\_cv\_n} and \texttt{std\_dev\_cv\_n}, where \texttt{n} ranges from 1 to $k$.
    \item \textbf{Average RMSE and Standard Deviation:} The overall cross-validation RMSE (\texttt{rmse\_cv}) and standard deviation (\texttt{std\_dev\_cv}) are computed as the mean of the fold-specific values.
    \item \textbf{Test Set RMSE and Standard Deviation:} The RMSE and standard deviation are also computed for the test set, denoted as \texttt{rmsep} and \texttt{std\_dev}, to provide a measure of the model's performance on unseen data.
\end{enumerate}

\subsubsection{Conclusion}
The implementation of a tailored validation and testing framework in this study ensures that the models developed are both accurate and generalizable.
By integrating custom k-fold cross-validation and carefully selecting performance metrics, the methodology effectively addresses potential issues of data leakage and overfitting.
These measures reinforce the reliability of the model evaluations and support the overarching goal of enhancing the accuracy and robustness of chemical composition analysis using \gls{libs} data.

% - Is it excessive to use both holdout and cross validation?
% - What are we missing here?
% - What is self-evident to us, but not to the reader?


\subsection{Evaluation Metrics}\label{subsec:evaluation_metrics}
To evaluate the performance of these models, we will use the \gls{rmse} to measure accuracy and the sample standard deviation of prediction errors to assess robustness.
We define accuracy as the ability of a model to predict the composition of major oxides in geological samples, while robustness refers to the stability of these predictions across samples.

The metric used to evaluate the accuracy of the models is the \gls{rmse}:
\[
\text{RMSE} = \sqrt{\frac{1}{n} \sum_{i=1}^{n} (\mathbf{v}_i - \hat{\mathbf{v}}_i)^2}
\]
where \( \mathbf{v}_i \) is the vector of actual oxide concentrations for the \( i \)-th sample, \( \hat{\mathbf{v}}_i \) is the corresponding vector of predicted oxide concentrations, and \( n \) is the total number of samples. 
This measure quantifies the average magnitude of the prediction error across all predicted values.

Robustness is evaluated using the sample standard deviation of prediction errors:
\[
\sigma_{error} = \sqrt{\frac{1}{n-1} \sum_{i=1}^{n} (e_i - \bar{e})^2}
\]
where \( e_i = \mathbf{v}_i - \hat{\mathbf{v}}_i \) and \( \bar{e} \) is the mean error.
A lower standard deviation indicates a more robust model across different samples.

These metrics are calculated for each fold and averaged across all folds to provide comprehensive indicators of model accuracy and variability.
In addition, we also compute the metrics for the test set to provide a measure of the model's performance on unseen data.
Therefore, we have the following metrics for each experiment:
\begin{enumerate}
    \item \textbf{Fold-specific \gls{rmse} and Standard Deviation:} For each of the $k$ folds, we calculate both the \gls{rmse} and standard deviation, denoted as \texttt{rmse\_cv\_n} and \texttt{std\_dev\_cv\_n}, where \texttt{n} ranges from 1 to $k$.
    \item \textbf{Average \gls{rmse} and Standard Deviation:} The overall cross-validation \gls{rmse} (\texttt{rmse\_cv}) and standard deviation (\texttt{std\_dev\_cv}) are computed as the mean of the fold-specific values. Formally, if \(\texttt{rmse\_cv\_n}\) and \(\texttt{std\_dev\_cv\_n}\) represent the \gls{rmse} and standard deviation for the \(n\)-th fold respectively, then:
    \[
    \texttt{rmse\_cv} = \frac{1}{k} \sum_{n=1}^{k} \texttt{rmse\_cv\_n}
    \]
    and
    \[
    \texttt{std\_dev\_cv} = \frac{1}{k} \sum_{n=1}^{k} \texttt{std\_dev\_cv\_n}
    \]
    where \(k\) is the total number of folds.
    \item \textbf{Test Set \gls{rmse} and Standard Deviation:} The \gls{rmse} and standard deviation are also computed for the test set, denoted as \texttt{rmsep} and \texttt{std\_dev}, to provide a measure of the model's performance on unseen data.
\end{enumerate}

K-fold cross-validation provides a robust estimate of model performance by averaging metrics over multiple folds, reducing variance and offering a clearer picture of the model's generalizability.
Evaluating the model on a separate test set representative of unseen data ensures that performance metrics accurately reflect the model's generalization capability.
However, our data partitioning method, which moves the most extreme values into the training data, naturally results in the testing data being closer to the mean of the data distribution, making it easier to predict.
In practice, this would result in lower \texttt{rmsep} and \texttt{std\_dev} values compared to the cross validation metrics.
Therefore, evaluating the model's performance using both the cross-validation metrics (\texttt{rmse\_cv} and \texttt{std\_dev\_cv}) and the test set metrics (\texttt{rmsep} and \texttt{std\_dev}) is crucial.
The cross-validation metrics provide insights into the model's stability across different subsets, while the test set metrics offer a final measure of performance on truly unseen data, giving a comprehensive assessment of the model's generalizability.

