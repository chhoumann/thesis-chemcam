\section{Methodology}\label{sec:methodology}
In this section, we outline the methodology used in this study to address the challenges identified in Section~\ref{sec:problem_definition}. Our objective was to identify the most promising machine learning models and preprocessing techniques proposed from the literature, as outlined in Section~\ref{sec:related-work}, for predicting major oxide compositions from \gls{libs} data.
Then, using this knowledge, develop a pipeline that utilizes the strengths of these models and preprocessing techniques to improve prediction accuracy and robustness of the predictions.
We first describe the datasets used, including their preparation and the method of splitting for model training. Next, we outline the preprocessing steps and the model selection process, followed by a detailed explanation of the experimental setup and evaluation metrics. Finally, we discuss our validation testing procedures and the approach taken to ensure unbiased final model evaluations.


\subsection{Data Preparation}
Similarly to our previous work \citet{p9_paper}, we used the publicly available \gls{ccs} data from NASA's \gls{pds}~\cite{PDSGeoscienceNode}.
\gls{ccs} refers to \gls{libs} data that has been through a series of preprocessing steps such as subtracting the ambient light background, noise removal and removing the electron continuum to derive data that is more suitable for quantitative analysis.
A comprehensive description of this preprocessing procedure is available in \citet{wiensPreflightCalibrationInitial2013}.

\begin{table*}[h]
\centering
\begin{tabular}{llllllll}
\toprule
     wave &         shot1 &         shot2 &  $\cdots$ &        shot49 &       shot50  & median        & mean          \\
\midrule
240.81100 & 6.4026649e+15 & 4.0429349e+15 & $\cdots$  & 1.7922483e+15 & 1.7126615e+15 & 1.9892956e+15 & 1.7561699e+15 \\
240.86501 & 3.8557462e+12 & 2.2923680e+12 & $\cdots$  & 1.1355429e+12 & 8.6930379e+11 & 7.8172542e+11 & 7.2805052e+11 \\
$\vdots$  & $\vdots$      & $\vdots$      & $\cdots$  & $\vdots$      & $\vdots$      & $\vdots$      & $\vdots$      \\
905.38062 & 1.8823427e+08 & 58500403.     & $\cdots$  & -8449286.6    & 8710775.0     & 4.0513312e+09 & 5.2188327e+09 \\
905.57349 & 1.9864713e+10 & 1.2956832e+10 & $\cdots$  & 1.9785415e+10 & 7.1994239e+09 & 1.1311150e+10 & 1.2201224e+10 \\
\bottomrule
\end{tabular}
\caption{Example of CCS data for a single location (from \citet{p9_paper})}
\label{tab:ccs_data_example}
\end{table*}

While the \gls{ccs} data is in a more suitable form for quantitative analysis, it still requires further preprocessing. Table~\ref{tab:ccs_data_example} shows an example of the \gls{ccs} data for a single location of a sample. This corresponds to shots ($s$) and wavelength ($\lambda$) of the Intensity Tensor \ref{matrix:intensity} for this location.
The initial five shots from each sample are excluded because they are usually contaminated by dust covering the sample, which is cleared away by the shock waves produced by the laser \cite{cleggRecalibrationMarsScience2017}.
The remaining 45 shots from each location are then averaged, yielding a single spectrum $s$ per location $l$ in the Averaged Intensity Tensor\ref{matrix:averaged_intensity}, resulting in a total of five spectra for each sample. 

At this stage, the data still contains noise at the edges of the spectrometers.
These edges correspond to the boundaries of the three spectrometers, which collectively cover the \gls{uv}, \gls{vio}, and \gls{vnir} light spectra.
The noisy edge ranges are as follows: 240.811-246.635 nm, 338.457-340.797 nm, 382.138-387.859 nm, 473.184-492.427 nm, and 849-905.574 nm.
In addition to being noisy regions, these regions do not contain any useful information related to each of the major oxides.
Consequently, these regions are masked by zeroing out the values, rather than removing them, as they represent meaningful variation in the data~\cite{cleggRecalibrationMarsScience2017}.

Additionally, as a result of the aforementioned preprocessing applied to the raw \gls{libs} data, negative values are present in the \gls{ccs} data.
These negative values are not physically meaningful, since you cannot have negative light intensity \cite{p9_paper}.
Similar to the noisy edges, these negative values are also masked by zeroing out the values.

We transpose the data so that each row represents a location and each column represents a wavelength feature. 
Each location is now represented as a vector of wavelengths, with the corresponding average intensity values for each wavelength. 
These vectors are then concatenated to form a tensor, giving us the full Averaged Intensity Tensor.

For each sample, we have a corresponding set of major oxide compositions in weight percentage (wt\%).
These compositions are used as the target labels for the machine learning models.
An excerpt of this data is shown in Table \ref{tab:composition_data_example}.
While the \textit{Target}, \textit{Spectrum Name}, and \textit{Sample Names} are part of the dataset, our analysis focuses primarily on the \textit{Sample Names}.
The concentrations of the eight oxides \ce{SiO2}, \ce{TiO2}, \ce{Al2O3}, \ce{FeO_T}, \ce{MnO}, \ce{MgO}, \ce{CaO}, \ce{Na2O}, and \ce{K2O} represent the expected values for these oxides in the sample, serving as our ground truth. The \textit{MOC total} is not utilized in this study.

\begin{table*}[h]
\centering
\begin{tabular}{lllllllllllll}
\toprule
     Target & Spectrum Name & Sample Name & \ce{SiO2} & \ce{TiO2} & \ce{Al2O3} & \ce{FeO_T} & \ce{MnO} & \ce{MgO} & \ce{CaO} & \ce{Na2O} & \ce{K2O} & \ce{MOC total} \\
\midrule
AGV2 & AGV2 & AGV2 & 59.3 & 1.05 & 16.91 & 6.02 & 0.099 & 1.79 & 5.2 & 4.19 & 2.88 & 97.44 \\
BCR-2 & BCR2 & BCR2 & 54.1 & 2.26 & 13.5 & 12.42 & 0.2 & 3.59 & 7.12 & 3.16 & 1.79 & 98.14 \\
$\vdots$ & $\vdots$ & $\vdots$ & $\vdots$ & $\vdots$ & $\vdots$ & $\vdots$ & $\vdots$ & $\vdots$ & $\vdots$ & $\vdots$ & $\vdots$ & $\vdots$ \\
TB & --- & --- & 60.23 & 0.93 & 20.64 & 11.6387 & 0.052 & 1.93 & 0.000031 & 1.32 & 3.87 & 100.610731 \\
    TB2 & --- & --- & 60.4 & 0.93 & 20.5 & 11.6536 & 0.047 & 1.86 & 0.2 & 1.29 & 3.86 & 100.7406 \\
\bottomrule
\end{tabular}
\caption{Excerpt from the composition dataset (from \citet{p9_paper})}
\label{tab:composition_data_example}
\end{table*}

The major oxide weight percentages are appended to the matrix of spectral data, forming the final dataset.
This dataset is shown in Table~\ref{tab:final_dataset_example}.
The \textit{Target} column corresponds to the sample name, while the \textit{ID} column contains the unique identifier for each location.

\begin{table*}[h]
\centering
\footnotesize
\begin{tabular}{llllllllllllllllllllll}
\toprule
    240.81   & $\cdots$     & 425.82    & 425.87   & $\cdots$ & 905.57  & \ce{SiO2} & \ce{TiO2} & \ce{Al2O3} & \ce{FeO_T} & \ce{MgO} & \ce{CaO} & \ce{Na2O} & \ce{K2O} & Target     & ID \\
\midrule
	0        & $\cdots$     & 1.53e+10 & 1.62e+10 & $\cdots$ & 0        & 56.13     & 0.69 & 17.69 & 5.86 & 3.85 & 7.07 & 3.32 & 1.44 & jsc1421     & jsc1421\_2013\_09\_12\_211002\_ccs \\
	0        & $\cdots$     & 1.28e+10 & 1.30e+10 & $\cdots$ & 0        & 56.13     & 0.69 & 17.69 & 5.86 & 3.85 & 7.07 & 3.32 & 1.44 & jsc1421     & jsc1421\_2013\_09\_12\_211143\_ccs \\
    0        & $\cdots$     & 1.87e+10 & 1.83e+10 & $\cdots$ & 0        & 56.13     & 0.69 & 17.69 & 5.86 & 3.85 & 7.07 & 3.32 & 1.44 & jsc1421     & jsc1421\_2013\_09\_12\_210628\_ccs \\
    0        & $\cdots$     & 1.77e+10 & 1.78e+10 & $\cdots$ & 0        & 56.13     & 0.69 & 17.69 & 5.86 & 3.85 & 7.07 & 3.32 & 1.44 & jsc1421     & jsc1421\_2013\_09\_12\_210415\_ccs \\
    0        & $\cdots$     & 1.75e+10 & 1.79e+10 & $\cdots$ & 0        & 56.13     & 0.69 & 17.69 & 5.86 & 3.85 & 7.07 & 3.32 & 1.44 & jsc1421     & jsc1421\_2013\_09\_12\_210811\_ccs \\
    0        & $\cdots$     & 5.52e+10 & 3.74e+10 & $\cdots$ & 0        & 57.60     & 0.78 & 26.60 & 2.73 & 0.70 & 0.01 & 0.38 & 7.10 & pg7         & pg7\_2013\_11\_07\_161903\_ccs \\
    0        & $\cdots$     & 5.09e+10 & 3.41e+10 & $\cdots$ & 0        & 57.60     & 0.78 & 26.60 & 2.73 & 0.70 & 0.01 & 0.38 & 7.10 & pg7         & pg7\_2013\_11\_07\_162038\_ccs \\
    0        & $\cdots$     & 5.99e+10 & 3.97e+10 & $\cdots$ & 0        & 57.60     & 0.78 & 26.60 & 2.73 & 0.70 & 0.01 & 0.38 & 7.10 & pg7         & pg7\_2013\_11\_07\_161422\_ccs \\
    0        & $\cdots$     & 5.22e+10 & 3.47e+10 & $\cdots$ & 0        & 57.60     & 0.78 & 26.60 & 2.73 & 0.70 & 0.01 & 0.38 & 7.10 & pg7         & pg7\_2013\_11\_07\_161735\_ccs \\
    0        & $\cdots$     & 5.29e+10 & 3.62e+10 & $\cdots$ & 0        & 57.60     & 0.78 & 26.60 & 2.73 & 0.70 & 0.01 & 0.38 & 7.10 & pg7         & pg7\_2013\_11\_07\_161552\_ccs \\
	$\vdots$ & $\cdots$ & $\vdots$ & $\vdots$ & $\cdots$ & $\vdots$ & $\vdots$ & $\vdots$ & $\vdots$ & $\vdots$ & $\vdots$ & $\vdots$ & $\vdots$ & $\vdots$ & $\vdots$ & $\vdots$ \\
\midrule
\end{tabular}
\caption{Excerpt from the final dataset (values have been rounded to two decimal places for brevity).}
\label{tab:final_dataset_example}
\end{table*}


\subsection{Model and Preprocessing Selection}
% Structure:
    % Introduction where we relate back to the problem def. We know these "general" themes can handle the challenges outlined in problem def.
    % Stacking ensemble is the approach we chose, therefore each item below is required to work with this..
    % Requirements for preprocessing:
        % We wanted to experiment with all the standard normalization methods (Z-score, minmax etc.).
        % We wanted to also try dimensionality reduction (PCA) - Why disqualify others(nice to have - perhaps didnt see in literature?
        % We discovered that there was lacking literature in the field of libs regarding transformers (power transform, quantile transform) and kernel pca - Therefore we decided to include these to assess their potential
            % Why these are relevant (refer back to background)
        % Feature selection was considered, but decided against due to time
    % Requirements for models:
        % We needed a variety of models to get a breadth of results
            % Each model should be different enough that it made sense to use in stacking ensemble
                % i.e. the internal approach/architecture should be different (meaning even though we selected many tree models for our final pipeline, they are still different enough to be useful in the ensemble)
        % They should be suitable for LIBS data as evidenced by the literature
            % Assessed by RMSEP/RMSECV - Whichever was available in the papers
        % .. But we also had some other models in minds
            % They should show potential at handling high dimensional data (this excludes i.e. straight up linear regression)
            % Should have potential to handle non-linear data
            % Should handle regression tasks
        % We did not care for computational efficiency or speed of prediction. Accuracy and robustness uber alles.
    % Requirements for model training strategies:
        % We considered several approaches, but settled on stacking ensemble due to its potential to combine the strengths of the models
        % Analogous to MOC (pasting/bagging (dont mention directly or at all) - average predictions), natural to try other approach in the same family (stacking - weighted average predictions)
        % We learned (from literature) that different models perform well on different oxides, so we wanted to leverage this

Choosing the right models and preprocessing techniques for \gls{libs} data analysis is a challenging task. 
The literature suggests a variety of models and preprocessing techniques that promise to be adept at handling data that exhibits high-dimensionality, non-linear relationships, multi-collinearity, and matrix effects.
Additionally, different machine learning models perform best on different oxides in LIBS data due to the unique spectral features and varying signal-to-noise ratios influenced by the physical properties of each oxide, such as atomic structure and ionization potential, which affect the emission lines detected.
To effectively address these challenges, one promising approach is to integrate multiple models and preprocessing techniques through stacking ensemble methods.
We adopted an experimental approach to empirically evaluate the potential of various models and preprocessing techniques, ensuring that our selections were informed by existing literature while also allowing for independent assessment and validation.

To guide our selection of preprocessing techniques, we had several considerations.
Firstly, our review of the literature revealed that there is no consensus on a single, most effective normalization method for \gls{libs} data.
This led us to include the traditional normalization methods, such as Z-score normalization, Min-Max scaling, and Max Absolute scaling, in our experiments.
This approach allowed us to determine which normalization method was most effective for our dataset. 
Additionally, dimensionality reduction techniques, such as \gls{pca}, were considered by the literature to be effective at reducing the dimensionality of the data while preserving the most important information. 
However, we found that there was a lack of literature on the effectiveness of other preprocessing techniques, such as power transformation, quantile transformation, and \gls{kernel-pca}, in the context of \gls{libs} data.

\subsection{Experimental Setup}
Experiments were conducted on a machine equipped with an Intel Xeon Gold 6242 CPU, featuring 16 cores and 32 threads.
The CPU has a base clock speed of 2.80 GHz and a maximum turbo frequency of 3.90 GHz.
The system has 64 GB of RAM and runs on Ubuntu 22.04.2 LTS.
Models were implemented using Python 3.10.11.
The primary libraries used were Scikit-learn 1.4.2, XGBoost 2.0.3, Torch 2.2.2, NumPy 1.26.4, Pandas 2.2.1, Keras 3.2.1 and Optuna 3.6.1.

% \subsection{Validation and Testing Procedures}
\subsection{Validation and Testing Procedures}
This section describes the validation and testing procedures that each of our experiments adhere to.
While we use conventional techniques like holdout sets and k-fold cross validation, \gls{libs} impose additional challenges to the process.

One of the primary challenges lies in ensuring there is no data leakage.
Therefore, we group shots on a given location for a target.
As per matrix A in Section~\ref{sec:problem_definition}, we know that each target only has one ground truth concentration value per oxide.
Since the emitted light intensities varies slightly for each location that is shot at on the target, we essentially have $l$ (locations per target) entries for the same ground truth value.
This means that if we randomly split the dataset, some locations from a target would end up in the testing set, while others in the training set.
This would result in data leakage, as the testing set would no longer be representative of the overall dataset.

% TODO: We'll be rewriting part of this due to rewriting our approach to splitting. We'll emulate the method employed by Anderson et al. completely, meaning we'll do this:
% > "To assess the performance of the PLS model, we use k-fold cross validation [55] and a separate test-set. We divided the full set of laboratory data into five folds, four of which are used for cross validation and are combined as the final training set. The fifth fold is withheld and used as a test set to validate the final model. Folds were defined separately for each element, after removal of outliers. To ensure that each fold represented the full elemental compositional variation, the samples were sorted based on the major element of interest, and samples were then assigned sequentially to each fold. In some cases, the database contains only a small number of targets with very high or very low compositions for a given element. These extreme targets can have a strong influence on the model and enable it to handle a wider range of compositions, so they are forced to be in the training folds rather than the test set. Unless otherwise specified, the results presented are the test set results."


\subsubsection{Dataset Partitioning}
To ensure a rigorous evaluation of the models, we first split the dataset into training and testing subsets.
As described in Section~\ref{sec:baseline_replica}, we adopted an automated approach to identify and distribute extreme compositions evenly across both subsets.
This method was devised to replicate the dataset splitting methodology employed by \citet{andersonImprovedAccuracyQuantitative2017}, as we have described in \citet{p9_paper}.
The process of dataset partitioning involves several key steps aimed at ensuring that extreme values are appropriately represented, which is critical given the skewness that such values can introduce into the training process:

\begin{enumerate}
    \item \textbf{Identification of Extremes:} For each oxide in the dataset, the samples are sorted by concentration. The extremes, defined as the $n$ highest and lowest values, are identified for inclusion in the training set.
    \item \textbf{Separation of Extremes:} These extreme samples are first segregated to ensure they are not included in the subsequent random partitioning. This step guarantees that the training set contains critical outliers which are often informative for model robustness.
    \item \textbf{Random Partitioning:} The remainder of the dataset, excluding the previously segregated extremes, undergoes a random split. The splitting process adheres to a predefined ratio, typically set at 80\%/20\% for training and testing, respectively.
    \item \textbf{Reintegration of Extremes:} The identified extremes are reintegrated into the training subset. This methodological step ensures that the training data encompasses a comprehensive range of the data's variability, particularly the tail-end characteristics that are crucial for robust model performance.
    \item \textbf{Adjustment of Test Size:} Given the inclusion of extremes in the training set, the proportion of the dataset allocated for testing is adjusted accordingly. This adjustment ensures that the overall ratio of training to testing remains as intended, despite the prior allocation of extreme samples to the training set.
\end{enumerate}

This approach not only aids in achieving a balanced dataset but also in maintaining the integrity of the testing process by avoiding any potential leakage of information between the training and testing datasets.

% Need to discuss our reasoning behind selecting n=2 for separating extreme values. Should be backed by data analysis ideally.

As previously discussed, we have opted for an 80\%/20\% division between the training and testing datasets.
This ratio is strategically chosen to maximize the training set's capacity for effective model learning while ensuring the testing set is sufficiently representative to provide an accurate assessment of the model's performance on new, unseen data.
Expanding the testing set beyond this proportion is not recommended.
As detailed in Section~\ref{sec:problem_definition}, one of the primary constraints we face is the limited availability of data.
Allocating too much data to the testing set could compromise the comprehensiveness of the training set, potentially undermining the model's ability to generalize from a robust learning process.


\subsubsection{Cross-Validation Strategy}
In this project, we implement a robust mechanism to prevent our model from merely tuning to peculiarities of a specific dataset segment.
Traditional approaches, where models are validated against a singular test set, might inadvertently result in models that perform well on that set but poorly generalize to new data.
To overcome this, we utilize a k-fold cross-validation strategy, which is particularly designed to enhance model generalizability across various unseen data scenarios.

Our strategy adopts a group-based variant of k-fold cross-validation to address potential data leakage, which can occur when closely related data points are scattered across both training and testing sets.
This can mislead the evaluation of the model's performance.
To mitigate this, our method ensures that all measurements related to a single entity (as defined by a grouping attribute) are contained entirely within either the training set or the testing set, but not both.

% What is a group?

The custom cross-validation method is delineated in Algorithm \ref{alg:custom_k_fold}: 

\begin{algorithm}[H]
\caption{Custom K-Fold Cross-Validation}
\label{alg:custom_k_fold}
\begin{algorithmic}[1]
\Require Dataset $D$, Number of folds $k$, Grouping attribute \textit{group\_by}, Random seed \textit{random\_state}
\Ensure Sequence of training and testing datasets for each fold

\State Group $D$ by \textit{group\_by} into $G$ \label{line:group}
\State Extract unique keys from $G$ into \textit{keys} \label{line:extract_keys}
\State Shuffle \textit{keys} using \textit{random\_state} \label{line:shuffle}
\State Split \textit{keys} into $k$ folds using K-Fold technique \label{line:split}
\For{$i = 1$ to $k$}
    \State Select the $i$-th fold as test keys, and the rest as train keys \label{line:select_keys}
    \State Concatenate groups corresponding to train keys to form \textit{train\_data} \label{line:concatenate_train}
    \State Concatenate groups corresponding to test keys to form \textit{test\_data} \label{line:concatenate_test}
    \State Yield $(\textit{train\_data}, \textit{test\_data})$ \label{line:yield}
\EndFor
\end{algorithmic}
\end{algorithm}

Initially, the dataset \(D\) is grouped by a specified attribute, resulting in groups \(G\) as described in line \ref{line:group}.
Unique group identifiers are extracted into \(keys\) (line \ref{line:extract_keys}), which are then shuffled (line \ref{line:shuffle}) to ensure randomness, utilizing a provided random seed \textit{random\_state}.

These keys are divided into \(k\) folds (line \ref{line:split}), and for each iteration from 1 to \(k\), one fold is selected as the test set, with the remaining serving as the training set (line \ref{line:select_keys}).
Corresponding data for each set of keys is then aggregated to form the training data (\textit{train\_data}) and the testing data (\textit{test\_data}), respectively, as indicated in lines \ref{line:concatenate_train} and \ref{line:concatenate_test}.
This ensures that all data from any single group is exclusively included in either the training or the testing set, thus mitigating the risk of data leakage.

This custom cross-validation strategy is crucial for ensuring that our evaluations are as realistic as possible, providing a reliable estimate of how the model will perform on truly independent data.
Through this method, we enhance the likelihood that our model's effectiveness is genuine and not a result of overfitting to the idiosyncrasies of the test data.


\subsubsection{Evaluation Metrics}
As mentioned in Section~\ref{sec:problem_definition}, the performance of the models was quantitatively assessed using the \gls{rmse} and the standard deviation of the residuals.
These metrics are calculated for each fold and averaged across all folds to provide comprehensive indicators of model accuracy and variability.
In addition, we also compute the metrics for the test set to provide a measure of the model's performance on unseen data.
Therefore, we have the following metrics for each experiment:
\begin{enumerate}
    \item \textbf{Fold-specific RMSE and Standard Deviation:} For each of the $k$ folds, we calculate both the RMSE and standard deviation, denoted as \texttt{rmse\_cv\_n} and \texttt{std\_dev\_cv\_n}, where \texttt{n} ranges from 1 to $k$.
    \item \textbf{Average RMSE and Standard Deviation:} The overall cross-validation RMSE (\texttt{rmse\_cv}) and standard deviation (\texttt{std\_dev\_cv}) are computed as the mean of the fold-specific values.
    \item \textbf{Test Set RMSE and Standard Deviation:} The RMSE and standard deviation are also computed for the test set, denoted as \texttt{rmsep} and \texttt{std\_dev}, to provide a measure of the model's performance on unseen data.
\end{enumerate}

\subsubsection{Conclusion}
The implementation of a tailored validation and testing framework in this study ensures that the models developed are both accurate and generalizable.
By integrating custom k-fold cross-validation and carefully selecting performance metrics, the methodology effectively addresses potential issues of data leakage and overfitting.
These measures reinforce the reliability of the model evaluations and support the overarching goal of enhancing the accuracy and robustness of chemical composition analysis using \gls{libs} data.

% - Is it excessive to use both holdout and cross validation?
% - What are we missing here?
% - What is self-evident to us, but not to the reader?




\subsection{Summary}
